\documentclass[10pt,landscape]{article}
\usepackage{multicol}
\usepackage{calc}
\usepackage{ifthen}
\usepackage[landscape]{geometry}
\usepackage{hyperref}
\usepackage{enumitem}
\usepackage{amsmath}

% To make this come out properly in landscape mode, do one of the following
% 1.
%  pdflatex latexsheet.tex
%
% 2.
%  latex latexsheet.tex
%  dvips -P pdf  -t landscape latexsheet.dvi
%  ps2pdf latexsheet.ps


% If you're reading this, be prepared for confusion.  Making this was
% a learning experience for me, and it shows.  Much of the placement
% was hacked in; if you make it better, let me know...


% 2008-04
% Changed page margin code to use the geometry package. Also added code for
% conditional page margins, depending on paper size. Thanks to Uwe Ziegenhagen
% for the suggestions.

% 2006-08
% Made changes based on suggestions from Gene Cooperman. <gene at ccs.neu.edu>


% To Do:
% \listoffigures \listoftables
% \setcounter{secnumdepth}{0}


% This sets page margins to .5 inch if using letter paper, and to 1cm
% if using A4 paper. (This probably isn't strictly necessary.)
% If using another size paper, use default 1cm margins.
\ifthenelse{\lengthtest { \paperwidth = 11in}}
	{ \geometry{top=.5in,left=.5in,right=.5in,bottom=.5in} }
	{\ifthenelse{ \lengthtest{ \paperwidth = 297mm}}
		{\geometry{top=1cm,left=1cm,right=1cm,bottom=1cm} }
		{\geometry{top=1cm,left=1cm,right=1cm,bottom=1cm} }
	}

% Turn off header and footer
\pagestyle{empty}
 

% Redefine section commands to use less space
\makeatletter
\renewcommand{\section}{\@startsection{section}{1}{0mm}%
                                {-1ex plus -.5ex minus -.2ex}%
                                {0.5ex plus .2ex}%x
                                {\normalfont\large\bfseries}}
\renewcommand{\subsection}{\@startsection{subsection}{2}{0mm}%
                                {-1explus -.5ex minus -.2ex}%
                                {0.5ex plus .2ex}%
                                {\normalfont\normalsize\bfseries}}
\renewcommand{\subsubsection}{\@startsection{subsubsection}{3}{0mm}%
                                {-1ex plus -.5ex minus -.2ex}%
                                {1ex plus .2ex}%
                                {\normalfont\small\bfseries}}
\makeatother

% Define BibTeX command
\def\BibTeX{{\rm B\kern-.05em{\sc i\kern-.025em b}\kern-.08em
    T\kern-.1667em\lower.7ex\hbox{E}\kern-.125emX}}

% Don't print section numbers
\setcounter{secnumdepth}{0}


\setlength{\parindent}{0pt}
\setlength{\parskip}{0pt plus 0.5ex}


% -----------------------------------------------------------------------

\begin{document}

\raggedright
\footnotesize
\begin{multicols}{3}


% multicol parameters
% These lengths are set only within the two main columns
%\setlength{\columnseprule}{0.25pt}
\setlength{\premulticols}{1pt}
\setlength{\postmulticols}{1pt}
\setlength{\multicolsep}{1pt}
\setlength{\columnsep}{2pt}

\begin{center}
     \Large{\textbf{Álgebra Linear}} \\
\end{center}

\section{Transformações lineares e matrizes Pt.1}

\hrule
\medskip

Condições de existência da transformação linear

\textbf{Def.} Sejam $U$ e $V$, espaços vetoriais reais, uma função $T:U\rightarrow V$ é uma \textbf{Transformação Linear} se:

\begin{enumerate}[label=(\roman*),wide=0pt]

\item Para quaisquer $u_1, u_2 \in U, T(u_1 + u_2) = T(u_1) + T(u_2)$
\item Dados $\beta \in R, u \in U, t(\beta\cdot u)=\beta\cdot T(u)$

\end{enumerate}

\textbf{Obs.} Quando $U=V$, $T$ é chamado um \textbf{operador linear} do espaço vetorial $U$.

\medskip 
\hrule
\medskip

\textbf{Prop.} Sejam $U$ e $V$ espaços vetoriais. Se $T:U\rightarrow V$ é uma transformação linear, então $T(\vec{0}_u)=\vec{0}_V$

Uma transformação linear sempre leva o vetor nulo de um espaço ao vetor nulo do outro. Se isso não acontecer, então não é uma transformação linear. No entando, o fato dessa condição ser satisfeita não garante que é uma transformação linear.

\medskip
\hrule
\medskip

Dado $u=(x_1,x_2,\ldots,x_n) \in R^n$, podemos associar o vetor $u$ a uma matriz coluna com $n$ linhas:

$$
u=
\begin{bmatrix}
x_1\\
x_2\\
\vdots\\
x_n
\end{bmatrix}
$$

e vice versa.

Um exemplo desse tipo de utilização é no caso de uma operação linear do tipo $T:R^2\rightarrow R^2$ definida como $T(x,y)=A\cdot\begin{bmatrix}x\\y\end{bmatrix}$ onde a matriz $A \in M_2(R)$ 

Toda aplicação de uma matriz multiplicada por um elemento do $R^n$ é uma aplicação linear.

\medskip
\hrule
\medskip

Toda função $T:R^n\rightarrow R^m$ dada por $T(x_1,x_2,\ldots,x_n)=A\cdot\begin{bmatrix}x_1\\x_2\\ \vdots\\ x_n\end{bmatrix}$ onde a matriz $A \in M_{m\times n}(R)$ é uma transformação linear.

\textbf{Prop.} Toda transformação linear $T:R^n\rightarrow R^m$ pode ser expressa na forma acima, para alguma matriz $A\in M_{m\times n}(R)$

\textbf{Exemplo} Seja $/T:R^3\rightarrow R^2$ dada por $T(x,y,z)=(x+y,x-z)$. Encontre uma matriz $A \in M_{2\times 3}(R)$ tal que $T$ se escreva na forma acima:

\medskip
$T(1,0,0)=(1, 1)$\\
$T(0,1,0)=(1, 0)$\\
$T(0,0,1)=(0,-1)$\\
\medskip
$A=\begin{bmatrix}1&1&0\\1&0&-1\end{bmatrix}$

\textbf{Def.} Esta matriz $A$ é chamada \textbf{matriz da transformação linear} $T$ em relação às bases canônicas de $R^3$ e $R^2$.

\textbf{Notação:} $A=[T]_{Can}$ ou simplesmente $A=[T]$

\medskip
\hrule
\medskip

Sejam $U$ e $V$ espaços vetoriais $dimU=n$ e $dimV=m$, $B$ e $F$ bases de $U$ e $V$ respectivamente, então para toda transformação linear $T:U\rightarrow V$ podemos encontrar uma matriz $A\in M_{m\times n}(R)$ tal que:

\medskip

$[T(u)]_F=A\cdot [u]_B, \forall u\in U$

\medskip

\textit{lê-se: as coordenadas da imagem do vetor $u$ na base $F$ é igual a matriz $A$ vezes o vetor $u$ na base $B$}

Onde $[T(u)]_F$ é a matriz coluna das coordenadas de $T(u)$ e $[u]_B$ é a matriz coluna das coordenadas de $u$ na base $B$. Notação: $A=[T]_{BF}$. $A$ é chamada matriz de $T$ com relação as bases $B$ e $F$

\textbf{Exemplo:} Seja $T:P_2(R)\rightarrow P_1(R)$, com $T(p)=p(0)+p(1)t$. Encontre a matriz de $T$ em relação às seguintes bases $B=\{1,t-1,t^2\}$ e $F=\{1,t-2\}$ respectivamente de $P_2(R)$ e $P_1(R)$

Vamos calcular $T$ nos vetores da base $B$.

$T(1)=1+T=(3,1)_F$ {\tiny primeiro na base B depois converte pra base F}\\
$T(t-1)=-1=(-1,0)_F$\\
$T(t^2)=1=(2,1)_F$

Portanto $A=[T_{BF}]=\begin{bmatrix}3&-1&2\\1&0&-1\end{bmatrix}$

Usando a matriz obtida acima calcule a imagem do polinômio $p(t)=2+3t-2t^2$

Temos $B=\{1,t-1,t^2\}$ e $F=\{1,t-2\}$

Vimos que $[T(p)]_F=[T]_BF\cdot[p]_B$

Primeiro convertemos o polinômio para a base $B$ $2+3t-2t=a+b(t-1)+ct^2$ portanto $a=5, b=3$ e $c=-2$ ou ainda $(5,3,-2)_B \Rightarrow [p]_B = \begin{bmatrix} 5\\3\\-2 \end{bmatrix}$, logo $[T(p)]_F=\begin{bmatrix}3&-1&2\\1&0&-1 \end{bmatrix}\cdot\begin{bmatrix} 5\\3\\-2 \end{bmatrix} = \begin{bmatrix} 8\\3 \end{bmatrix}$, assim, convertendo para a base $F$ temos $T(p)=(8,3)_F=8\cdot(1)+3\cdot(t-2)=2+3t$

\section{Transformações lineares e matrizes Pt.2}
 
\hrule
\medskip

\textbf{Exemplo} Seja $T:P_3{R}\rightarrow P_2{R}$ dada por $T(p)=p'$. Encontre a matriz de $T$ em relação às bases canônicas.

Sejam $B=\{1,t,t^2,t^3\}$ e $F=\{1,t,t^2\}$ as bases canônicas de $P_3(R)$ e $P_2{R}$ respectivamente.

Primeiro vamos calcular $T$ usando os vetores da base $B$

$T(1)=1{'}=0=(0,0,0)_F$\\
$T(t)=t{'}=1=(1,0,0)_F$\\
$T(t^2)=t^2{'}=2t=(0,2,0)_F$\\
$T(t^2)=t^3{'}=3t^2=(0,0,3)_F$

Portanto $[T]=\begin{bmatrix}0&1&0&0\\0&0&2&0\\0&0&0&3 \end{bmatrix}$

Vamos verificar o uso da matriz obtida no exemplo para calcular a derivada do polinômio $p(t)=2+3t-2t^2+4t^3$

Lembrando que $[T(p)]_F=[T]_{BF}\cdot [p]_B$ e que $p=(2,3,-2,4)\Rightarrow [p]_B=\begin{bmatrix}2\\3\\-2\\4 \end{bmatrix}$

$[T(p)]_F=\begin{bmatrix}0&1&0&0\\0&0&2&0\\0&0&0&3 \end{bmatrix}\cdot\begin{bmatrix}2\\3\\-2\\4 \end{bmatrix}=\begin{bmatrix}3\\-4\\12\end{bmatrix}$

Logo $T(p)=(3,-4,12)_F\Leftrightarrow p'(t)=3\cdot 1-4\cdot t+12\cdot t^2$

\medskip 
\hrule
\medskip

\textbf{Def.} Seja $U$ um espaço vetorial, o operador linear $I:U\rightarrow U$ dado por $I(u)=u, \forall u\in U$ será chamado operador identidade de $U$.

Se $B$ é uma base de $U$ então a matriz de $I$ com relação a base $B$ é a matriz identidade de ordem $n=dimU$ ou seja
\begin{center}
$\begin{bmatrix} 1&\cdots &0 \\ \vdots&\ddots&\vdots \\ 0&\cdots &1 \end{bmatrix}=I_n$
\end{center}

Seja $F$ uma outra base de $U$ então a matriz do operador $I$ com relação às bases $B$ e $F$ tem a seguinte propriedade: 
$[I(u)]_F=[I]_{BF}\cdot[u]B$, como $I(u)=u$ a expressão acima se transforma em: $[u]_F=[I]_{BF}\cdot[u]_B$

Assim, a matriz $[I]_{BF}$ relaciona as coordenadas do vetor $u$ na base $B$ com suas coordenadas na base $F$

\textbf{Exemplo:} Sejam $B={(1,0)(0,1)}$ e $F={(1,1), (1,-1)}$ bases de $R^2$. Determine a matriz do operador identidade em relação às bases $B$ e $F$. Se $u=(2,3)_B$ quais são suas coordenadas na base $F$?

$I(1,0)=(1,0)=\frac{1}{2}\cdot(1,1)+\frac{1}{2}(1,-1)=\left(\frac{1}{2},\frac{1}{2}\right)_F$\\
$I(0,1)=(0,1)=\frac{1}{2}\cdot(1,1)-\frac{1}{2}(1,-1)=\left(\frac{1}{2},-\frac{1}{2}\right)_F$\\

Portanto a matriz $[I]_{BF}=\begin{bmatrix} \frac{1}{2}&\frac{1}{2}\\ \frac{1}{2}&-\frac{1}{2}\end{bmatrix}=
\frac{1}{2}\cdot\begin{bmatrix} 1&1 \\ 1&-1 \end{bmatrix}$

$[u]_F=[I]_{BF}\cdot [u]_B$, $u=(2,3)\Rightarrow [u]_B=\begin{bmatrix} 2\\3 \end{bmatrix}$ 

Logo, $[u]_F=\frac{1}{2}\cdot\begin{bmatrix} 1&1 \\ 1&-1 \end{bmatrix}\cdot \begin{bmatrix} 2\\3 \end{bmatrix}=\frac{1}{2}\cdot \begin{bmatrix} 5\\-1 \end{bmatrix}\Rightarrow u = \left(\frac{5}{2},-\frac{1}{2}\right)_F$

\medskip 
\hrule
\medskip

Sejam $T$ um operador linear de um espaço vetorial $U$ e $B$ uma base qualquer de $U$. Então $T$ é invertível se, e só se, a matriz $[T]_B$ for invertível, neste caso $T^{-1}$ é um operador linear e vale:

$[T^{-1}]_B=([T]_B)^{-1}$

\textbf{Exemplo:} Seja $T:R^3\rightarrow R^3$ o operador linear dado por $T(x,y,z)=(2x+y,x-z,y+2z)$, $T$ é invertível? 

$T(1,0,0)=(2,1,0)$\\
$T(0,1,0)=(1,0,1)$\\
$T(0,0,1)=(0,-1,2)$

$[T]=\begin{bmatrix}2&1&0\\1&0&-1\\0&1&2\end{bmatrix},det[T]=2-2=0$ Para ser invertível o determinante tem que ser diferente de zero. Neste caso, $T$ não é invertível.

\textbf{Exemplo:} Seja $T:R^2\rightarrow R^2$ o operador linear dado por $T(x,y)=(x-2y,2x+y)$. Determine a matriz de $T^{-1}$ em relação à base canônica de $R^2$.

$T(1,0)=(1,2)$\\
$T(0,1)=(-2,1)$

$[T]=\begin{bmatrix}1&-2\\2&1\end{bmatrix}, det[T]=5\neq 0$, Logo $T$ é invertível

Para achar a matriz inversa $M_{2x2}$, basta pegar $\frac{1}{det}$ e multiplicar pela matriz da seguinte forma: trocar os elementos da diagonal principal e trocar o sinal da diagonal secundária. Portanto:

$[T^{-1}]=[T]^{-1}=\frac{1}{5}\cdot \begin{bmatrix}1&2\\-2&1\end{bmatrix}$

\textbf{Exemplo: } Calcule $T^{-1}(2,5)$ onde $T$ é o operador do exemplo anterior.

$[T^{-1}(2,5)]=[T]^{-1}\cdot[(2,5)], [(2,5)]=\begin{bmatrix}2\\5\end{bmatrix}$, portanto:

$[T^{-1}(2,5)]=\frac{1}{5}\cdot \begin{bmatrix}1&2\\-2&1\end{bmatrix}\cdot\begin{bmatrix}2\\5\end{bmatrix}=\frac{1}{5}\cdot\begin{bmatrix}12\\1\end{bmatrix}$, logo $T^{-1}(2,5)=(\frac{12}{5},\frac{1}{5})$

Para verificar fazemos $T\left(\frac{12}{5},\frac{1}{5}\right)=\frac{1}{5}T(12,1)=\frac{1}{5}(10,25)=(2,5)$

\medskip 
\hrule
\medskip

Sejam $T$ e $S$ operadores lineares de um espaço vetorial $U$ e $B$ uma base qualquer de $U$ então:

\begin{enumerate}[label=(\roman*),wide=0pt]

\item Dado $\beta \in R, \beta \cdot T:U\rightarrow U$ é linear e $[\beta\cdot T]_B=\beta[T]_B$
\item $T+S:U\rightarrow U$ é linear e $[T+S]_B=[T]_B+[S]_B$
\item $T\circ S:U\rightarrow U$ é linear e $[T\circ S]_B=[T]_B\cdot [S]_B$

\end{enumerate}

\textbf{Exemplo: } Sejam $T,S:R^2\rightarrow R^2$ operadores lineares definidos por $T(x,y)=(2x,x+y)$ e $S(x,y)=(y-x,3x)$.

Determine a matriz do operador $G=2T-3(T\circ S)$ em relação a base canônica de $R^2$

$[G]=[2T-3(T\circ S)]=2[T]-3[T][S]$

$T(1,0)=(2,1), T(0,1)=(0,1)$ logo $[T]=\begin{pmatrix}2&0\\1&1\end{pmatrix}$\\
$S(1,0)=(-1,3), S(0,1)=(1,0)$ logo $[S]=\begin{pmatrix}-1&1\\3&0\end{pmatrix}$

Portanto $[G]=2[T]-3[T][S]=2\cdot\begin{pmatrix}2&0\\1&1\end{pmatrix}-3\cdot\begin{pmatrix}2&0\\1&1\end{pmatrix}\cdot\begin{pmatrix}-1&1\\3&0\end{pmatrix}$\\

Logo $[G]=\begin{bmatrix}10&0\\-4&1\end{bmatrix}$

\textbf{Exemplo: } Sejam $T,I:R^2\rightarrow R^2$ operadores linares tais que $[T]=\begin{bmatrix}2&1\\3&-1\end{bmatrix}$ e $I(x,y)=(x,y)$

Determine a matriz do operador $G=T-3I$ em relação à base canônica de $R^2$.

$[G]=[T-3I]=[T]-3[I]=[T]-3I_2$

Assim $[G]=\begin{bmatrix}2&1\\3&-1\end{bmatrix}-3\begin{bmatrix}1&0\\0&1\end{bmatrix}=\begin{bmatrix}-1&1\\3&-4\end{bmatrix}$

\section{Núcleo e Imagem}

\hrule
\medskip

\begin{tabular}{@{}ll@{}}
\verb!book!    & Default is two-sided. \\
\verb!report!  & No \verb!\part! divisions. \\
\verb!article! & No \verb!\part! or \verb!\chapter! divisions. \\
\verb!letter!  & Letter (?). \\
\verb!slides!  & Large sans-serif font.
\end{tabular}

Used at the very beginning of a document:
\verb!\documentclass{!\textit{class}\verb!}!.  Use
\verb!\begin{document}! to start contents and \verb!\end{document}! to
end the document.


\subsection{Common \texttt{documentclass} options}
\newlength{\MyLen}
\settowidth{\MyLen}{\texttt{letterpaper}/\texttt{a4paper} \ }
\begin{tabular}{@{}p{\the\MyLen}%
                @{}p{\linewidth-\the\MyLen}@{}}
\texttt{10pt}/\texttt{11pt}/\texttt{12pt} & Font size. \\
\texttt{letterpaper}/\texttt{a4paper} & Paper size. \\
\texttt{twocolumn} & Use two columns. \\
\texttt{twoside}   & Set margins for two-sided. \\
\texttt{landscape} & Landscape orientation.  Must use
                     \texttt{dvips -t landscape}. \\
\texttt{draft}     & Double-space lines.
\end{tabular}

Usage:
\verb!\documentclass[!\textit{opt,opt}\verb!]{!\textit{class}\verb!}!.


\subsection{Packages}
\settowidth{\MyLen}{\texttt{multicol} }
\begin{tabular}{@{}p{\the\MyLen}%
                @{}p{\linewidth-\the\MyLen}@{}}
%\begin{tabular}{@{}ll@{}}
\texttt{fullpage}  &  Use 1 inch margins. \\
\texttt{anysize}   &  Set margins: \verb!\marginsize{!\textit{l}%
                        \verb!}{!\textit{r}\verb!}{!\textit{t}%
                        \verb!}{!\textit{b}\verb!}!.            \\
\texttt{multicol}  &  Use $n$ columns: 
                        \verb!\begin{multicols}{!$n$\verb!}!.   \\
\texttt{latexsym}  &  Use \LaTeX\ symbol font. \\
\texttt{graphicx}  &  Show image:
                        \verb!\includegraphics[width=!%
                        \textit{x}\verb!]{!%
                        \textit{file}\verb!}!. \\
\texttt{url}       & Insert URL: \verb!\url{!%
                        \textit{http://\ldots}%
                        \verb!}!.
\end{tabular}

Use before \verb!\begin{document}!. 
Usage: \verb!\usepackage{!\textit{package}\verb!}!


\subsection{Title}
\settowidth{\MyLen}{\texttt{.author.text.} }
\begin{tabular}{@{}p{\the\MyLen}%
                @{}p{\linewidth-\the\MyLen}@{}}
\verb!\author{!\textit{text}\verb!}! & Author of document. \\
\verb!\title{!\textit{text}\verb!}!  & Title of document. \\
\verb!\date{!\textit{text}\verb!}!   & Date. \\
\end{tabular}

These commands go before \verb!\begin{document}!.  The declaration
\verb!\maketitle! goes at the top of the document.

\subsection{Miscellaneous}
\settowidth{\MyLen}{\texttt{.pagestyle.empty.} }
\begin{tabular}{@{}p{\the\MyLen}%
                @{}p{\linewidth-\the\MyLen}@{}}
\verb!\pagestyle{empty}!     &   Empty header, footer
                                 and no page numbers. \\
\verb!\tableofcontents!      &   Add a table of contents here. \\

\end{tabular}



\section{Document structure}
\begin{multicols}{2}
\verb!\part{!\textit{title}\verb!}!  \\
\verb!\chapter{!\textit{title}\verb!}!  \\
\verb!\section{!\textit{title}\verb!}!  \\
\verb!\subsection{!\textit{title}\verb!}!  \\
\verb!\subsubsection{!\textit{title}\verb!}!  \\
\verb!\paragraph{!\textit{title}\verb!}!  \\
\verb!\subparagraph{!\textit{title}\verb!}!
\end{multicols}
{\raggedright
Use \verb!\setcounter{secnumdepth}{!$x$\verb!}! suppresses heading
numbers of depth $>x$, where \verb!chapter! has depth 0.
Use a \texttt{*}, as in \verb!\section*{!\textit{title}\verb!}!,
to not number a particular item---these items will also not appear
in the table of contents.
}

\subsection{Text environments}
\settowidth{\MyLen}{\texttt{.begin.quotation.}}
\begin{tabular}{@{}p{\the\MyLen}%
                @{}p{\linewidth-\the\MyLen}@{}}
\verb!\begin{comment}!    &  Comment (not printed). Requires \texttt{verbatim} package. \\
\verb!\begin{quote}!      &  Indented quotation block. \\
\verb!\begin{quotation}!  &  Like \texttt{quote} with indented paragraphs. \\
\verb!\begin{verse}!      &  Quotation block for verse.
\end{tabular}

\subsection{Lists}
\settowidth{\MyLen}{\texttt{.begin.description.}}
\begin{tabular}{@{}p{\the\MyLen}%
                @{}p{\linewidth-\the\MyLen}@{}}
\verb!\begin{enumerate}!        &  Numbered list. \\
\verb!\begin{itemize}!          &  Bulleted list. \\
\verb!\begin{description}!      &  Description list. \\
\verb!\item! \textit{text}      &  Add an item. \\
\verb!\item[!\textit{x}\verb!]! \textit{text}
                                &  Use \textit{x} instead of normal
                        bullet or number.  Required for descriptions. \\
\end{tabular}




\subsection{References}
\settowidth{\MyLen}{\texttt{.pageref.marker..}}
\begin{tabular}{@{}p{\the\MyLen}%
                @{}p{\linewidth-\the\MyLen}@{}}
\verb!\label{!\textit{marker}\verb!}!   &  Set a marker for cross-reference, 
                          often of the form \verb!\label{sec:item}!. \\
\verb!\ref{!\textit{marker}\verb!}!   &  Give section/body number of marker. \\
\verb!\pageref{!\textit{marker}\verb!}! &  Give page number of marker. \\
\verb!\footnote{!\textit{text}\verb!}!  &  Print footnote at bottom of page. \\
\end{tabular}


\subsection{Floating bodies}
\settowidth{\MyLen}{\texttt{.begin.equation..place.}}
\begin{tabular}{@{}p{\the\MyLen}%
                @{}p{\linewidth-\the\MyLen}@{}}
\verb!\begin{table}[!\textit{place}\verb!]!     &  Add numbered table. \\
\verb!\begin{figure}[!\textit{place}\verb!]!    &  Add numbered figure. \\
\verb!\begin{equation}[!\textit{place}\verb!]!  &  Add numbered equation. \\
\verb!\caption{!\textit{text}\verb!}!           &  Caption for the body. \\
\end{tabular}

The \textit{place} is a list valid placements for the body.  \texttt{t}=top,
\texttt{h}=here, \texttt{b}=bottom, \texttt{p}=separate page, \texttt{!}=place even if ugly.  Captions and label markers should be within the environment.

%---------------------------------------------------------------------------

\section{Text properties}

\subsection{Font face}
\newcommand{\FontCmd}[3]{\PBS\verb!\#1{!\textit{text}\verb!}!  \> %
                         \verb!{\#2 !\textit{text}\verb!}! \> %
                         \#1{#3}}
\begin{tabular}{@{}l@{}l@{}l@{}}
\textit{Command} & \textit{Declaration} & \textit{Effect} \\
\verb!\textrm{!\textit{text}\verb!}!                    & %
        \verb!{\rmfamily !\textit{text}\verb!}!               & %
        \textrm{Roman family} \\
\verb!\textsf{!\textit{text}\verb!}!                    & %
        \verb!{\sffamily !\textit{text}\verb!}!               & %
        \textsf{Sans serif family} \\
\verb!\texttt{!\textit{text}\verb!}!                    & %
        \verb!{\ttfamily !\textit{text}\verb!}!               & %
        \texttt{Typewriter family} \\
\verb!\textmd{!\textit{text}\verb!}!                    & %
        \verb!{\mdseries !\textit{text}\verb!}!               & %
        \textmd{Medium series} \\
\verb!\textbf{!\textit{text}\verb!}!                    & %
        \verb!{\bfseries !\textit{text}\verb!}!               & %
        \textbf{Bold series} \\
\verb!\textup{!\textit{text}\verb!}!                    & %
        \verb!{\upshape !\textit{text}\verb!}!               & %
        \textup{Upright shape} \\
\verb!\textit{!\textit{text}\verb!}!                    & %
        \verb!{\itshape !\textit{text}\verb!}!               & %
        \textit{Italic shape} \\
\verb!\textsl{!\textit{text}\verb!}!                    & %
        \verb!{\slshape !\textit{text}\verb!}!               & %
        \textsl{Slanted shape} \\
\verb!\textsc{!\textit{text}\verb!}!                    & %
        \verb!{\scshape !\textit{text}\verb!}!               & %
        \textsc{Small Caps shape} \\
\verb!\emph{!\textit{text}\verb!}!                      & %
        \verb!{\em !\textit{text}\verb!}!               & %
        \emph{Emphasized} \\
\verb!\textnormal{!\textit{text}\verb!}!                & %
        \verb!{\normalfont !\textit{text}\verb!}!       & %
        \textnormal{Document font} \\
\verb!\underline{!\textit{text}\verb!}!                 & %
                                                        & %
        \underline{Underline}
\end{tabular}

The command (t\textit{tt}t) form handles spacing better than the
declaration (t{\itshape tt}t) form.

\subsection{Font size}
\setlength{\columnsep}{14pt} % Need to move columns apart a little
\begin{multicols}{2}
\begin{tabbing}
\verb!\footnotesize!          \= \kill
\verb!\tiny!                  \>  \tiny{tiny} \\
\verb!\scriptsize!            \>  \scriptsize{scriptsize} \\
\verb!\footnotesize!          \>  \footnotesize{footnotesize} \\
\verb!\small!                 \>  \small{small} \\
\verb!\normalsize!            \>  \normalsize{normalsize} \\
\verb!\large!                 \>  \large{large} \\
\verb!\Large!                 \=  \Large{Large} \\  % Tab hack for new column
\verb!\LARGE!                 \>  \LARGE{LARGE} \\
\verb!\huge!                  \>  \huge{huge} \\
\verb!\Huge!                  \>  \Huge{Huge}
\end{tabbing}
\end{multicols}
\setlength{\columnsep}{1pt} % Set column separation back

These are declarations and should be used in the form
\verb!{\small! \ldots\verb!}!, or without braces to affect the entire
document.


\subsection{Verbatim text}

\settowidth{\MyLen}{\texttt{.begin.verbatim..} }
\begin{tabular}{@{}p{\the\MyLen}%
                @{}p{\linewidth-\the\MyLen}@{}}
\verb@\begin{verbatim}@ & Verbatim environment. \\
\verb@\begin{verbatim*}@ & Spaces are shown as \verb*@ @. \\
\verb@\verb!text!@ & Text between the delimiting characters (in this case %
                      `\texttt{!}') is verbatim.
\end{tabular}


\subsection{Justification}
\begin{tabular}{@{}ll@{}}
\textit{Environment}  &  \textit{Declaration}  \\
\verb!\begin{center}!      & \verb!\centering!  \\
\verb!\begin{flushleft}!  & \verb!\raggedright! \\
\verb!\begin{flushright}! & \verb!\raggedleft!  \\
\end{tabular}

\subsection{Miscellaneous}
\verb!\linespread{!$x$\verb!}! changes the line spacing by the
multiplier $x$.





\section{Text-mode symbols}

\subsection{Symbols}
\begin{tabular}{@{}l@{\hspace{1em}}l@{\hspace{2em}}l@{\hspace{1em}}l@{\hspace{2em}}l@{\hspace{1em}}l@{\hspace{2em}}l@{\hspace{1em}}l@{}}
\&              &  \verb!\&! &
\_              &  \verb!\_! &
\ldots          &  \verb!\ldots! &
\textbullet     &  \verb!\textbullet! \\
\$              &  \verb!\$! &
\^{}            &  \verb!\^{}! &
\textbar        &  \verb!\textbar! &
\textbackslash  &  \verb!\textbackslash! \\
\%              &  \verb!\%! &
\~{}            &  \verb!\~{}! &
\#              &  \verb!\#! &
\S              &  \verb!\S! \\
\end{tabular}

\subsection{Accents}
\begin{tabular}{@{}l@{\ }l|l@{\ }l|l@{\ }l|l@{\ }l|l@{\ }l@{}}
\`o   & \verb!\`o! &
\'o   & \verb!\'o! &
\^o   & \verb!\^o! &
\~o   & \verb!\~o! &
\=o   & \verb!\=o! \\
\.o   & \verb!\.o! &
\"o   & \verb!\"o! &
\c o  & \verb!\c o! &
\v o  & \verb!\v o! &
\H o  & \verb!\H o! \\
\c c  & \verb!\c c! &
\d o  & \verb!\d o! &
\b o  & \verb!\b o! &
\t oo & \verb!\t oo! &
\oe   & \verb!\oe! \\
\OE   & \verb!\OE! &
\ae   & \verb!\ae! &
\AE   & \verb!\AE! &
\aa   & \verb!\aa! &
\AA   & \verb!\AA! \\
\o    & \verb!\o! &
\O    & \verb!\O! &
\l    & \verb!\l! &
\L    & \verb!\L! &
\i    & \verb!\i! \\
\j    & \verb!\j! &
!`    & \verb!~`! &
?`    & \verb!?`! &
\end{tabular}


\subsection{Delimiters}
\begin{tabular}{@{}l@{\ }ll@{\ }ll@{\ }ll@{\ }ll@{\ }ll@{\ }l@{}}
`       & \verb!`!  &
``      & \verb!``! &
\{      & \verb!\{! &
\lbrack & \verb![! &
(       & \verb!(! &
\textless  &  \verb!\textless! \\
'       & \verb!'!  &
''      & \verb!''! &
\}      & \verb!\}! &
\rbrack & \verb!]! &
)       & \verb!)! &
\textgreater  &  \verb!\textgreater! \\
\end{tabular}

\subsection{Dashes}
\begin{tabular}{@{}llll@{}}
\textit{Name} & \textit{Source} & \textit{Example} & \textit{Usage} \\
hyphen  & \verb!-!   & X-ray          & In words. \\
en-dash & \verb!--!  & 1--5           & Between numbers. \\
em-dash & \verb!---! & Yes---or no?    & Punctuation.
\end{tabular}


\subsection{Line and page breaks}
\settowidth{\MyLen}{\texttt{.pagebreak} }
\begin{tabular}{@{}p{\the\MyLen}%
                @{}p{\linewidth-\the\MyLen}@{}}
\verb!\\!          &  Begin new line without new paragraph.  \\
\verb!\\*!         &  Prohibit pagebreak after linebreak. \\
\verb!\kill!       &  Don't print current line. \\
\verb!\pagebreak!  &  Start new page. \\
\verb!\noindent!   &  Do not indent current line.
\end{tabular}


\subsection{Miscellaneous}
\settowidth{\MyLen}{\texttt{.rule.w..h.} }
\begin{tabular}{@{}p{\the\MyLen}%
                @{}p{\linewidth-\the\MyLen}@{}}
\verb!\today!  &  \today. \\
\verb!$\sim$!  &  Prints $\sim$ instead of \verb!\~{}!, which makes \~{}. \\
\verb!~!       &  Space, disallow linebreak (\verb!W.J.~Clinton!). \\
\verb!\@.!     &  Indicate that the \verb!.! ends a sentence when following
                        an uppercase letter. \\
\verb!\hspace{!$l$\verb!}! & Horizontal space of length $l$
                                (Ex: $l=\mathtt{20pt}$). \\
\verb!\vspace{!$l$\verb!}! & Vertical space of length $l$. \\
\verb!\rule{!$w$\verb!}{!$h$\verb!}! & Line of width $w$ and height $h$. \\
\end{tabular}



\section{Tabular environments}

\subsection{\texttt{tabbing} environment}
\begin{tabular}{@{}l@{\hspace{1.5ex}}l@{\hspace{10ex}}l@{\hspace{1.5ex}}l@{}}
\verb!\=!  &   Set tab stop. &
\verb!\>!  &   Go to tab stop.
\end{tabular}

Tab stops can be set on ``invisible'' lines with \verb!\kill!
at the end of the line.  Normally \verb!\\! is used to separate lines.


\subsection{\texttt{tabular} environment}
\verb!\begin{array}[!\textit{pos}\verb!]{!\textit{cols}\verb!}!   \\
\verb!\begin{tabular}[!\textit{pos}\verb!]{!\textit{cols}\verb!}! \\
\verb!\begin{tabular*}{!\textit{width}\verb!}[!\textit{pos}\verb!]{!\textit{cols}\verb!}!


\subsubsection{\texttt{tabular} column specification}
\settowidth{\MyLen}{\texttt{p}\{\textit{width}\} \ }
\begin{tabular}{@{}p{\the\MyLen}@{}p{\linewidth-\the\MyLen}@{}}
\texttt{l}    &   Left-justified column.  \\
\texttt{c}    &   Centered column.  \\
\texttt{r}    &   Right-justified column. \\
\verb!p{!\textit{width}\verb!}!  &  Same as %
                              \verb!\parbox[t]{!\textit{width}\verb!}!. \\ 
\verb!@{!\textit{decl}\verb!}!   &  Insert \textit{decl} instead of
                                    inter-column space. \\
\verb!|!      &   Inserts a vertical line between columns. 
\end{tabular}


\subsubsection{\texttt{tabular} elements}
\settowidth{\MyLen}{\texttt{.cline.x-y..}}
\begin{tabular}{@{}p{\the\MyLen}@{}p{\linewidth-\the\MyLen}@{}}
\verb!\hline!           &  Horizontal line between rows.  \\
\verb!\cline{!$x$\verb!-!$y$\verb!}!  &
                        Horizontal line across columns $x$ through $y$. \\
\verb!\multicolumn{!\textit{n}\verb!}{!\textit{cols}\verb!}{!\textit{text}\verb!}! \\
        &  A cell that spans \textit{n} columns, with \textit{cols} column specification.
\end{tabular}

\section{Math mode}
For inline math, use \verb!\(...\)! or \verb!$...$!.
For displayed math, use \verb!\[...\]! or \verb!\begin{equation}!.

\begin{tabular}{@{}l@{\hspace{1em}}l@{\hspace{2em}}l@{\hspace{1em}}l@{}}
Superscript$^{x}$       &
\verb!^{x}!             &  
Subscript$_{x}$         &
\verb!_{x}!             \\  
$\frac{x}{y}$           &
\verb!\frac{x}{y}!      &  
$\sum_{k=1}^n$          &
\verb!\sum_{k=1}^n!     \\  
$\sqrt[n]{x}$           &
\verb!\sqrt[n]{x}!      &  
$\prod_{k=1}^n$         &
\verb!\prod_{k=1}^n!    \\ 
\end{tabular}

\subsection{Math-mode symbols}

% The ordering of these symbols is slightly odd.  This is because I had to put all the
% long pieces of text in the same column (the right) for it all to fit properly.
% Otherwise, it wouldn't be possible to fit four columns of symbols here.

\begin{tabular}{@{}l@{\hspace{1ex}}l@{\hspace{1em}}l@{\hspace{1ex}}l@{\hspace{1em}}l@{\hspace{1ex}} l@{\hspace{1em}}l@{\hspace{1ex}}l@{}}
$\leq$          &  \verb!\leq!  &
$\geq$          &  \verb!\geq!  &
$\neq$          &  \verb!\neq!  &
$\approx$       &  \verb!\approx!  \\
$\times$        &  \verb!\times!  &
$\div$          &  \verb!\div!  &
$\pm$           & \verb!\pm!  &
$\cdot$         &  \verb!\cdot!  \\
$^{\circ}$      & \verb!^{\circ}! &
$\circ$         &  \verb!\circ!  &
$\prime$        & \verb!\prime!  &
$\cdots$        &  \verb!\cdots!  \\
$\infty$        & \verb!\infty!  &
$\neg$          & \verb!\neg!  &
$\wedge$        & \verb!\wedge!  &
$\vee$          & \verb!\vee!  \\
$\supset$       & \verb!\supset!  &
$\forall$       & \verb!\forall!  &
$\in$           & \verb!\in!  &
$\rightarrow$   &  \verb!\rightarrow! \\
$\subset$       & \verb!\subset!  &
$\exists$       & \verb!\exists!  &
$\notin$        & \verb!\notin!  &
$\Rightarrow$   &  \verb!\Rightarrow! \\
$\cup$          & \verb!\cup!  &
$\cap$          & \verb!\cap!  &
$\mid$          & \verb!\mid!  &
$\Leftrightarrow$   &  \verb!\Leftrightarrow! \\
$\dot a$        & \verb!\dot a!  &
$\hat a$        & \verb!\hat a!  &
$\bar a$        & \verb!\bar a!  &
$\tilde a$      & \verb!\tilde a!  \\

$\alpha$        &  \verb!\alpha!  &
$\beta$         &  \verb!\beta!  &
$\gamma$        &  \verb!\gamma!  &
$\delta$        &  \verb!\delta!  \\
$\epsilon$      &  \verb!\epsilon!  &
$\zeta$         &  \verb!\zeta!  &
$\eta$          &  \verb!\eta!  &
$\varepsilon$   &  \verb!\varepsilon!  \\
$\theta$        &  \verb!\theta!  &
$\iota$         &  \verb!\iota!  &
$\kappa$        &  \verb!\kappa!  &
$\vartheta$     &  \verb!\vartheta!  \\
$\lambda$       &  \verb!\lambda!  &
$\mu$           &  \verb!\mu!  &
$\nu$           &  \verb!\nu!  &
$\xi$           &  \verb!\xi!  \\
$\pi$           &  \verb!\pi!  &
$\rho$          &  \verb!\rho!  &
$\sigma$        &  \verb!\sigma!  &
$\tau$          &  \verb!\tau!  \\
$\upsilon$      &  \verb!\upsilon!  &
$\phi$          &  \verb!\phi!  &
$\chi$          &  \verb!\chi!  &
$\psi$          &  \verb!\psi!  \\
$\omega$        &  \verb!\omega!  &
$\Gamma$        &  \verb!\Gamma!  &
$\Delta$        &  \verb!\Delta!  &
$\Theta$        &  \verb!\Theta!  \\
$\Lambda$       &  \verb!\Lambda!  &
$\Xi$           &  \verb!\Xi!  &
$\Pi$           &  \verb!\Pi!  &
$\Sigma$        &  \verb!\Sigma!  \\
$\Upsilon$      &  \verb!\Upsilon!  &
$\Phi$          &  \verb!\Phi!  &
$\Psi$          &  \verb!\Psi!  &
$\Omega$        &  \verb!\Omega!  
\end{tabular}
\footnotesize

%\subsection{Special symbols}
%\begin{tabular}{@{}ll@{}}
%$^{\circ}$  &  \verb!^{\circ}! Ex: $22^{\circ}\mathrm{C}$: \verb!$22^{\circ}\mathrm{C}$!.
%\end{tabular}

\section{Bibliography and citations}
When using \BibTeX, you need to run \texttt{latex}, \texttt{bibtex},
and \texttt{latex} twice more to resolve dependencies.

\subsection{Citation types}
\settowidth{\MyLen}{\texttt{.shortciteN.key..}}
\begin{tabular}{@{}p{\the\MyLen}@{}p{\linewidth-\the\MyLen}@{}}
\verb!\cite{!\textit{key}\verb!}!       &
        Full author list and year. (Watson and Crick 1953) \\
\verb!\citeA{!\textit{key}\verb!}!      &
        Full author list. (Watson and Crick) \\
\verb!\citeN{!\textit{key}\verb!}!      &
        Full author list and year. Watson and Crick (1953) \\
\verb!\shortcite{!\textit{key}\verb!}!  &
        Abbreviated author list and year. ? \\
\verb!\shortciteA{!\textit{key}\verb!}! &
        Abbreviated author list. ? \\
\verb!\shortciteN{!\textit{key}\verb!}! &
        Abbreviated author list and year. ? \\
\verb!\citeyear{!\textit{key}\verb!}!   &
        Cite year only. (1953) \\
\end{tabular}

All the above have an \texttt{NP} variant without parentheses;
Ex. \verb!\citeNP!.


\subsection{\BibTeX\ entry types}
\settowidth{\MyLen}{\texttt{.mastersthesis.}}
\begin{tabular}{@{}p{\the\MyLen}@{}p{\linewidth-\the\MyLen}@{}}
\verb!@article!         &  Journal or magazine article. \\
\verb!@book!            &  Book with publisher. \\
\verb!@booklet!         &  Book without publisher. \\
\verb!@conference!      &  Article in conference proceedings. \\
\verb!@inbook!          &  A part of a book and/or range of pages. \\
\verb!@incollection!    &  A part of book with its own title. \\
%\verb!@manual!          &  Technical documentation. \\
%\verb!@mastersthesis!   &  Master's thesis. \\
\verb!@misc!            &  If nothing else fits. \\
\verb!@phdthesis!       &  PhD. thesis. \\
\verb!@proceedings!     &  Proceedings of a conference. \\
\verb!@techreport!      &  Tech report, usually numbered in series. \\
\verb!@unpublished!     &  Unpublished. \\
\end{tabular}

\subsection{\BibTeX\ fields}
\settowidth{\MyLen}{\texttt{organization.}}
\begin{tabular}{@{}p{\the\MyLen}@{}p{\linewidth-\the\MyLen}@{}}
\verb!address!         &  Address of publisher.  Not necessary for major
                                publishers.  \\
\verb!author!           &  Names of authors, of format .... \\
\verb!booktitle!        &  Title of book when part of it is cited. \\
\verb!chapter!          &  Chapter or section number. \\
\verb!edition!          &  Edition of a book. \\
\verb!editor!           &  Names of editors. \\
\verb!institution!      &  Sponsoring institution of tech.\ report. \\
\verb!journal!          &  Journal name. \\
\verb!key!              &  Used for cross ref.\ when no author. \\
\verb!month!            &  Month published. Use 3-letter abbreviation. \\
\verb!note!             &  Any additional information. \\
\verb!number!           &  Number of journal or magazine. \\
\verb!organization!     &  Organization that sponsors a conference. \\
\verb!pages!            &  Page range (\verb!2,6,9--12!). \\
\verb!publisher!        &  Publisher's name. \\
\verb!school!           &  Name of school (for thesis). \\
\verb!series!           &  Name of series of books. \\
\verb!title!            &  Title of work. \\
\verb!type!             &  Type of tech.\ report, ex. ``Research Note''. \\
\verb!volume!           &  Volume of a journal or book. \\
\verb!year!             &  Year of publication. \\
\end{tabular}
Not all fields need to be filled.  See example below.

\subsection{Common \BibTeX\ style files}
\begin{tabular}{@{}l@{\hspace{1em}}l@{\hspace{3em}}l@{\hspace{1em}}l@{}}
\verb!abbrv!    &  Standard &
\verb!abstract! &  \texttt{alpha} with abstract \\
\verb!alpha!    &  Standard &
\verb!apa!      &  APA \\
\verb!plain!    &  Standard &
\verb!unsrt!    &  Unsorted \\
\end{tabular}

The \LaTeX\ document should have the following two lines just before
\verb!\end{document}!, where \verb!bibfile.bib! is the name of the
\BibTeX\ file.
\begin{verbatim}
\bibliographystyle{plain}
\bibliography{bibfile}
\end{verbatim}

\subsection{\BibTeX\ example}
The \BibTeX\ database goes in a file called
\textit{file}\texttt{.bib}, which is processed with \verb!bibtex file!. 
\begin{verbatim}
@String{N = {Na\-ture}}
@Article{WC:1953,
  author  = {James Watson and Francis Crick},
  title   = {A structure for Deoxyribose Nucleic Acid},
  journal = N,
  volume  = {171},
  pages   = {737},
  year    = 1953
}
\end{verbatim}


\section{Sample \LaTeX\ document}
\begin{verbatim}
\documentclass[11pt]{article}
\usepackage{fullpage}
\title{Template}
\author{Name}
\begin{document}
\maketitle

\section{section}
\subsection*{subsection without number}
text \textbf{bold text} text. Some math: $2+2=5$
\subsection{subsection}
text \emph{emphasized text} text. \cite{WC:1953}
discovered the structure of DNA.

A table:
\begin{table}[!th]
\begin{tabular}{|l|c|r|}
\hline
first  &  row  &  data \\
second &  row  &  data \\
\hline
\end{tabular}
\caption{This is the caption}
\label{ex:table}
\end{table}

The table is numbered \ref{ex:table}.
\end{document}
\end{verbatim}



\rule{0.3\linewidth}{0.25pt}
\scriptsize

Copyright \copyright\ 2019 Márcio de Souza Maia Junior

\href{http://wch.github.io/latexsheet/}{http://wch.github.io/latexsheet/}


\end{multicols}
\end{document}
